% Options for packages loaded elsewhere
\PassOptionsToPackage{unicode}{hyperref}
\PassOptionsToPackage{hyphens}{url}
%
\documentclass[
]{article}
\usepackage{amsmath,amssymb}
\usepackage{iftex}
\ifPDFTeX
  \usepackage[T1]{fontenc}
  \usepackage[utf8]{inputenc}
  \usepackage{textcomp} % provide euro and other symbols
\else % if luatex or xetex
  \usepackage{unicode-math} % this also loads fontspec
  \defaultfontfeatures{Scale=MatchLowercase}
  \defaultfontfeatures[\rmfamily]{Ligatures=TeX,Scale=1}
\fi
\usepackage{lmodern}
\ifPDFTeX\else
  % xetex/luatex font selection
\fi
% Use upquote if available, for straight quotes in verbatim environments
\IfFileExists{upquote.sty}{\usepackage{upquote}}{}
\IfFileExists{microtype.sty}{% use microtype if available
  \usepackage[]{microtype}
  \UseMicrotypeSet[protrusion]{basicmath} % disable protrusion for tt fonts
}{}
\makeatletter
\@ifundefined{KOMAClassName}{% if non-KOMA class
  \IfFileExists{parskip.sty}{%
    \usepackage{parskip}
  }{% else
    \setlength{\parindent}{0pt}
    \setlength{\parskip}{6pt plus 2pt minus 1pt}}
}{% if KOMA class
  \KOMAoptions{parskip=half}}
\makeatother
\usepackage{xcolor}
\usepackage[margin=1in]{geometry}
\usepackage{color}
\usepackage{fancyvrb}
\newcommand{\VerbBar}{|}
\newcommand{\VERB}{\Verb[commandchars=\\\{\}]}
\DefineVerbatimEnvironment{Highlighting}{Verbatim}{commandchars=\\\{\}}
% Add ',fontsize=\small' for more characters per line
\usepackage{framed}
\definecolor{shadecolor}{RGB}{248,248,248}
\newenvironment{Shaded}{\begin{snugshade}}{\end{snugshade}}
\newcommand{\AlertTok}[1]{\textcolor[rgb]{0.94,0.16,0.16}{#1}}
\newcommand{\AnnotationTok}[1]{\textcolor[rgb]{0.56,0.35,0.01}{\textbf{\textit{#1}}}}
\newcommand{\AttributeTok}[1]{\textcolor[rgb]{0.13,0.29,0.53}{#1}}
\newcommand{\BaseNTok}[1]{\textcolor[rgb]{0.00,0.00,0.81}{#1}}
\newcommand{\BuiltInTok}[1]{#1}
\newcommand{\CharTok}[1]{\textcolor[rgb]{0.31,0.60,0.02}{#1}}
\newcommand{\CommentTok}[1]{\textcolor[rgb]{0.56,0.35,0.01}{\textit{#1}}}
\newcommand{\CommentVarTok}[1]{\textcolor[rgb]{0.56,0.35,0.01}{\textbf{\textit{#1}}}}
\newcommand{\ConstantTok}[1]{\textcolor[rgb]{0.56,0.35,0.01}{#1}}
\newcommand{\ControlFlowTok}[1]{\textcolor[rgb]{0.13,0.29,0.53}{\textbf{#1}}}
\newcommand{\DataTypeTok}[1]{\textcolor[rgb]{0.13,0.29,0.53}{#1}}
\newcommand{\DecValTok}[1]{\textcolor[rgb]{0.00,0.00,0.81}{#1}}
\newcommand{\DocumentationTok}[1]{\textcolor[rgb]{0.56,0.35,0.01}{\textbf{\textit{#1}}}}
\newcommand{\ErrorTok}[1]{\textcolor[rgb]{0.64,0.00,0.00}{\textbf{#1}}}
\newcommand{\ExtensionTok}[1]{#1}
\newcommand{\FloatTok}[1]{\textcolor[rgb]{0.00,0.00,0.81}{#1}}
\newcommand{\FunctionTok}[1]{\textcolor[rgb]{0.13,0.29,0.53}{\textbf{#1}}}
\newcommand{\ImportTok}[1]{#1}
\newcommand{\InformationTok}[1]{\textcolor[rgb]{0.56,0.35,0.01}{\textbf{\textit{#1}}}}
\newcommand{\KeywordTok}[1]{\textcolor[rgb]{0.13,0.29,0.53}{\textbf{#1}}}
\newcommand{\NormalTok}[1]{#1}
\newcommand{\OperatorTok}[1]{\textcolor[rgb]{0.81,0.36,0.00}{\textbf{#1}}}
\newcommand{\OtherTok}[1]{\textcolor[rgb]{0.56,0.35,0.01}{#1}}
\newcommand{\PreprocessorTok}[1]{\textcolor[rgb]{0.56,0.35,0.01}{\textit{#1}}}
\newcommand{\RegionMarkerTok}[1]{#1}
\newcommand{\SpecialCharTok}[1]{\textcolor[rgb]{0.81,0.36,0.00}{\textbf{#1}}}
\newcommand{\SpecialStringTok}[1]{\textcolor[rgb]{0.31,0.60,0.02}{#1}}
\newcommand{\StringTok}[1]{\textcolor[rgb]{0.31,0.60,0.02}{#1}}
\newcommand{\VariableTok}[1]{\textcolor[rgb]{0.00,0.00,0.00}{#1}}
\newcommand{\VerbatimStringTok}[1]{\textcolor[rgb]{0.31,0.60,0.02}{#1}}
\newcommand{\WarningTok}[1]{\textcolor[rgb]{0.56,0.35,0.01}{\textbf{\textit{#1}}}}
\usepackage{graphicx}
\makeatletter
\def\maxwidth{\ifdim\Gin@nat@width>\linewidth\linewidth\else\Gin@nat@width\fi}
\def\maxheight{\ifdim\Gin@nat@height>\textheight\textheight\else\Gin@nat@height\fi}
\makeatother
% Scale images if necessary, so that they will not overflow the page
% margins by default, and it is still possible to overwrite the defaults
% using explicit options in \includegraphics[width, height, ...]{}
\setkeys{Gin}{width=\maxwidth,height=\maxheight,keepaspectratio}
% Set default figure placement to htbp
\makeatletter
\def\fps@figure{htbp}
\makeatother
\setlength{\emergencystretch}{3em} % prevent overfull lines
\providecommand{\tightlist}{%
  \setlength{\itemsep}{0pt}\setlength{\parskip}{0pt}}
\setcounter{secnumdepth}{-\maxdimen} % remove section numbering
\ifLuaTeX
  \usepackage{selnolig}  % disable illegal ligatures
\fi
\IfFileExists{bookmark.sty}{\usepackage{bookmark}}{\usepackage{hyperref}}
\IfFileExists{xurl.sty}{\usepackage{xurl}}{} % add URL line breaks if available
\urlstyle{same}
\hypersetup{
  pdftitle={4. Scoping Rules},
  hidelinks,
  pdfcreator={LaTeX via pandoc}}

\title{4. Scoping Rules}
\author{}
\date{\vspace{-2.5em}}

\begin{document}
\maketitle

\hypertarget{scoping-rules}{%
\section{4. Scoping Rules}\label{scoping-rules}}

\hypertarget{scoping-rules---symbol-binding}{%
\subsection{Scoping Rules - Symbol
Binding}\label{scoping-rules---symbol-binding}}

\hypertarget{a-diversion-on-binding-values-to-symbol}{%
\subsubsection{A Diversion on Binding Values to
Symbol}\label{a-diversion-on-binding-values-to-symbol}}

How does R know which value to assign to which symbol? When I type

\begin{Shaded}
\begin{Highlighting}[]
\NormalTok{lm }\OtherTok{\textless{}{-}} \ControlFlowTok{function}\NormalTok{(x) \{}
\NormalTok{  x }\SpecialCharTok{*}\NormalTok{ x}
\NormalTok{\}}

\NormalTok{lm}
\end{Highlighting}
\end{Shaded}

\begin{verbatim}
## function(x) {
##   x * x
## }
\end{verbatim}

how does R know what value to assign to the symbol lm? Why doesn't it
give it the value of lm that is in the stats package?

When R tries to bind a value to a symbol, it searches through a series
of environments to find the appropriate value. When you are working on
the command line and need to retrieve the value of an R object, the
order is roughly 1. Search the global environment for a symbol name
matching the one requested.\\
2. Search the namespaces of each of the packages on the search list The
search list can be found by using the search function.

\begin{Shaded}
\begin{Highlighting}[]
\FunctionTok{search}\NormalTok{()}
\end{Highlighting}
\end{Shaded}

\begin{verbatim}
## [1] ".GlobalEnv"        "package:stats"     "package:graphics" 
## [4] "package:grDevices" "package:utils"     "package:datasets" 
## [7] "package:methods"   "Autoloads"         "package:base"
\end{verbatim}

\hypertarget{binding-values-to-symbol}{%
\subsubsection{Binding Values to
Symbol}\label{binding-values-to-symbol}}

· The global environment or the user's workspace is always the first
element of the search list and the base package is always the last. ·
The order of the packages on the search list matters! · User's can
configure which packages get loaded on startup so you cannot assume that
there will be a set list of packages available. · When a user loads a
package with library the namespace of that package gets put in position
2 of the search list (by default) and everything else gets shifted down
the list. · Note that R has separate namespaces for functions and
non-functions so it's possible to have an object named c and a function
named c.

\hypertarget{scoping-rules-1}{%
\subsubsection{Scoping Rules}\label{scoping-rules-1}}

The scoping rules for R are the main feature that make it different from
the original S language. · The scoping rules determine how a value is
associated with a free variable in a function · R uses lexical scoping
or static scoping. A common alternative is dynamic scoping. · Related to
the scoping rules is how R uses the search list to bind a value to a
symbol · Lexical scoping turns out to be particularly useful for
simplifying statistical computations

\hypertarget{lexical-scoping}{%
\subsubsection{Lexical Scoping}\label{lexical-scoping}}

Consider the following function.

\begin{Shaded}
\begin{Highlighting}[]
\NormalTok{f }\OtherTok{\textless{}{-}} \ControlFlowTok{function}\NormalTok{(x, y) \{}
\NormalTok{  x }\SpecialCharTok{\^{}} \DecValTok{2} \SpecialCharTok{+}\NormalTok{ y }\SpecialCharTok{/}\NormalTok{ z}
\NormalTok{\}}
\end{Highlighting}
\end{Shaded}

This function has 2 formal arguments x and y. In the body of the
function there is another symbol z. In this case z is called a free
variable. The scoping rules of a language determine how values are
assigned to free variables. Free variables are not formal arguments and
are not local variables (assigned insided the function body).

Lexical scoping in R means that the values of free variables are
searched for in the environment in which the function was defined. What
is an environment? · An environment is a collection of (symbol, value)
pairs, i.e.~x is a symbol and 3.14 might be its value. · Every
environment has a parent environment; it is possible for an environment
to have multiple ``children'' · the only environment without a parent is
the empty environment · A function + an environment = a closure or
function closure.

Searching for the value for a free variable: · If the value of a symbol
is not found in the environment in which a function was defined, then
the search is continued in the parent environment. · The search
continues down the sequence of parent environments until we hit the
top-level environment; this usually the global environment (workspace)
or the namespace of a package. · After the top-level environment, the
search continues down the search list until we hit the empty
environment. If a value for a given symbol cannot be found once the
empty environment is arrived at, then an error is thrown.

\hypertarget{scoping-rules---r-scoping-rules}{%
\subsection{Scoping Rules - R Scoping
Rules}\label{scoping-rules---r-scoping-rules}}

\hypertarget{lexical-scoping-1}{%
\subsubsection{Lexical Scoping}\label{lexical-scoping-1}}

Why does all this matter? · Typically, a function is defined in the
global environment, so that the values of free variables are just found
in the user's workspace · This behavior is logical for most people and
is usually the ``right thing'' to do · However, in R you can have
functions defined inside other functions -
LanguageslikeCdon'tletyoudothis · Now things get interesting --- In this
case the environment in which a function is defined is the body of
another function!

\begin{Shaded}
\begin{Highlighting}[]
\NormalTok{make.power }\OtherTok{\textless{}{-}} \ControlFlowTok{function}\NormalTok{(n) \{}
\NormalTok{  pow }\OtherTok{\textless{}{-}} \ControlFlowTok{function}\NormalTok{(x) \{}
\NormalTok{    x }\SpecialCharTok{\^{}}\NormalTok{ n}
\NormalTok{  \}}
\NormalTok{  pow}
\NormalTok{\}}
\end{Highlighting}
\end{Shaded}

This function returns another function as its value.

\begin{Shaded}
\begin{Highlighting}[]
\NormalTok{cube }\OtherTok{\textless{}{-}} \FunctionTok{make.power}\NormalTok{(}\DecValTok{3}\NormalTok{)}
\NormalTok{square }\OtherTok{\textless{}{-}} \FunctionTok{make.power}\NormalTok{(}\DecValTok{2}\NormalTok{)}
\end{Highlighting}
\end{Shaded}

\begin{Shaded}
\begin{Highlighting}[]
\FunctionTok{cube}\NormalTok{(}\DecValTok{3}\NormalTok{)}
\end{Highlighting}
\end{Shaded}

\begin{verbatim}
## [1] 27
\end{verbatim}

\begin{Shaded}
\begin{Highlighting}[]
\FunctionTok{square}\NormalTok{(}\DecValTok{2}\NormalTok{)}
\end{Highlighting}
\end{Shaded}

\begin{verbatim}
## [1] 4
\end{verbatim}

\hypertarget{exploring-a-function-closure}{%
\subsubsection{Exploring a Function
Closure}\label{exploring-a-function-closure}}

What's in a function's environment?

\begin{Shaded}
\begin{Highlighting}[]
\FunctionTok{ls}\NormalTok{(}\FunctionTok{environment}\NormalTok{(cube))}
\end{Highlighting}
\end{Shaded}

\begin{verbatim}
## [1] "n"   "pow"
\end{verbatim}

\begin{Shaded}
\begin{Highlighting}[]
\FunctionTok{get}\NormalTok{(}\StringTok{"n"}\NormalTok{, }\FunctionTok{environment}\NormalTok{(cube))}
\end{Highlighting}
\end{Shaded}

\begin{verbatim}
## [1] 3
\end{verbatim}

\begin{Shaded}
\begin{Highlighting}[]
\FunctionTok{ls}\NormalTok{(}\FunctionTok{environment}\NormalTok{(square))}
\end{Highlighting}
\end{Shaded}

\begin{verbatim}
## [1] "n"   "pow"
\end{verbatim}

\begin{Shaded}
\begin{Highlighting}[]
\FunctionTok{get}\NormalTok{(}\StringTok{"n"}\NormalTok{, }\FunctionTok{environment}\NormalTok{(square))}
\end{Highlighting}
\end{Shaded}

\begin{verbatim}
## [1] 2
\end{verbatim}

\hypertarget{lexical-vs.-dynamic-scoping}{%
\subsubsection{Lexical vs.~Dynamic
Scoping}\label{lexical-vs.-dynamic-scoping}}

\begin{Shaded}
\begin{Highlighting}[]
\NormalTok{y }\OtherTok{\textless{}{-}} \DecValTok{10}
\NormalTok{f }\OtherTok{\textless{}{-}} \ControlFlowTok{function}\NormalTok{(x) \{}
\NormalTok{  y }\OtherTok{\textless{}{-}} \DecValTok{2}
\NormalTok{  y }\SpecialCharTok{\^{}} \DecValTok{2} \SpecialCharTok{+} \FunctionTok{g}\NormalTok{(x)}
\NormalTok{\}}
\NormalTok{g }\OtherTok{\textless{}{-}} \ControlFlowTok{function}\NormalTok{(x) \{}
\NormalTok{  x }\SpecialCharTok{*}\NormalTok{ y}
\NormalTok{\}}
\end{Highlighting}
\end{Shaded}

\begin{Shaded}
\begin{Highlighting}[]
\NormalTok{y}
\end{Highlighting}
\end{Shaded}

\begin{verbatim}
## [1] 10
\end{verbatim}

\begin{Shaded}
\begin{Highlighting}[]
\FunctionTok{f}\NormalTok{(}\DecValTok{3}\NormalTok{)}
\end{Highlighting}
\end{Shaded}

\begin{verbatim}
## [1] 34
\end{verbatim}

· With lexical scoping the value of y in the function g is looked up in
the environment in which the function was defined, in this case the
global environment, so the value of y is 10. · With dynamic scoping, the
value of y is looked up in the environment from which the function was
called (sometimes referred to as the calling environment). -
InRthecallingenvironmentisknownastheparentframe · So the value ofywould
be 2.

When a function is defined in the global environment and is subsequently
called from the global environment, then the defining environment and
the calling environment are the same. This can sometimes give the
appearance of dynamic scoping.

\begin{Shaded}
\begin{Highlighting}[]
\NormalTok{g }\OtherTok{\textless{}{-}} \ControlFlowTok{function}\NormalTok{(x) \{}
\NormalTok{  a }\OtherTok{\textless{}{-}} \DecValTok{3}
\NormalTok{  x }\SpecialCharTok{+}\NormalTok{ a }\SpecialCharTok{+}\NormalTok{ y}
\NormalTok{\}}
\end{Highlighting}
\end{Shaded}

\begin{Shaded}
\begin{Highlighting}[]
\FunctionTok{g}\NormalTok{(}\DecValTok{2}\NormalTok{)}
\end{Highlighting}
\end{Shaded}

\begin{verbatim}
## [1] 15
\end{verbatim}

\begin{Shaded}
\begin{Highlighting}[]
\NormalTok{y}\OtherTok{\textless{}{-}}\DecValTok{3}
\FunctionTok{g}\NormalTok{(}\DecValTok{2}\NormalTok{)}
\end{Highlighting}
\end{Shaded}

\begin{verbatim}
## [1] 8
\end{verbatim}

\hypertarget{other-languages}{%
\subsubsection{Other Languages}\label{other-languages}}

Other languages that support lexical scoping · Scheme · Perl · Python ·
Common Lisp (all languages converge to Lisp)

\hypertarget{consequences-of-lexical-scoping}{%
\subsubsection{Consequences of Lexical
Scoping}\label{consequences-of-lexical-scoping}}

In R, all objects must be stored in memory · All functions must carry a
pointer to their respective defining environments, which could be
anywhere · In S-PLUS, free variables are always looked up in the global
workspace, so everything can be stored on the disk because the
``defining environment'' of all functions is the same.

\hypertarget{scoping-rules---optimization-example-optional}{%
\subsection{Scoping Rules - Optimization Example
(OPTIONAL)}\label{scoping-rules---optimization-example-optional}}

\hypertarget{application-optimization}{%
\subsubsection{Application:
Optimization}\label{application-optimization}}

Why is any of this information useful? · Optimization routines in R like
optim, nlm, and optimize require you to pass a function whose argument
is a vector of parameters (e.g.~a log-likelihood) · However, an object
function might depend on a host of other things besides its parameters
(like data) · When writing software which does optimization, it may be
desirable to allow the user to hold certain parameters fixed

\hypertarget{maximizing-a-normal-likelihood}{%
\subsubsection{Maximizing a Normal
Likelihood}\label{maximizing-a-normal-likelihood}}

Write a ``constructor'' function

\begin{Shaded}
\begin{Highlighting}[]
\NormalTok{make.NegLogLik }\OtherTok{\textless{}{-}}
  \ControlFlowTok{function}\NormalTok{(data, }\AttributeTok{fixed =} \FunctionTok{c}\NormalTok{(}\ConstantTok{FALSE}\NormalTok{, }\ConstantTok{FALSE}\NormalTok{)) \{}
\NormalTok{    params }\OtherTok{\textless{}{-}}\NormalTok{ fixed}
    \ControlFlowTok{function}\NormalTok{(p) \{}
\NormalTok{      params[}\SpecialCharTok{!}\NormalTok{fixed] }\OtherTok{\textless{}{-}}\NormalTok{ p}
\NormalTok{      mu }\OtherTok{\textless{}{-}}\NormalTok{ params[}\DecValTok{1}\NormalTok{]}
\NormalTok{      sigma }\OtherTok{\textless{}{-}}\NormalTok{ params[}\DecValTok{2}\NormalTok{]}
\NormalTok{      a }\OtherTok{\textless{}{-}} \SpecialCharTok{{-}}\FloatTok{0.5} \SpecialCharTok{*} \FunctionTok{length}\NormalTok{(data) }\SpecialCharTok{*} \FunctionTok{log}\NormalTok{(}\DecValTok{2} \SpecialCharTok{*}\NormalTok{ pi }\SpecialCharTok{*}\NormalTok{ sigma }\SpecialCharTok{\^{}} \DecValTok{2}\NormalTok{) }
\NormalTok{      b }\OtherTok{\textless{}{-}} \SpecialCharTok{{-}}\FloatTok{0.5} \SpecialCharTok{*} \FunctionTok{sum}\NormalTok{((data }\SpecialCharTok{{-}}\NormalTok{ mu) }\SpecialCharTok{\^{}} \DecValTok{2}\NormalTok{) }\SpecialCharTok{/}\NormalTok{ (sigma }\SpecialCharTok{\^{}} \DecValTok{2}\NormalTok{) }
      \SpecialCharTok{{-}}\NormalTok{ (a }\SpecialCharTok{+}\NormalTok{ b)}
\NormalTok{    \}}
\NormalTok{  \}}
\end{Highlighting}
\end{Shaded}

Note: Optimization functions in R minimize functions, so you need to use
the negative log-likelihood.

\hypertarget{maximizing-a-normal-likelihood-1}{%
\subsubsection{Maximizing a Normal
Likelihood}\label{maximizing-a-normal-likelihood-1}}

\begin{Shaded}
\begin{Highlighting}[]
\FunctionTok{set.seed}\NormalTok{(}\DecValTok{1}\NormalTok{); normals }\OtherTok{\textless{}{-}} \FunctionTok{rnorm}\NormalTok{(}\DecValTok{100}\NormalTok{, }\DecValTok{1}\NormalTok{, }\DecValTok{2}\NormalTok{)}
\NormalTok{nLL }\OtherTok{\textless{}{-}} \FunctionTok{make.NegLogLik}\NormalTok{(normals)}
\NormalTok{nLL}
\end{Highlighting}
\end{Shaded}

\begin{verbatim}
## function(p) {
##       params[!fixed] <- p
##       mu <- params[1]
##       sigma <- params[2]
##       a <- -0.5 * length(data) * log(2 * pi * sigma ^ 2) 
##       b <- -0.5 * sum((data - mu) ^ 2) / (sigma ^ 2) 
##       - (a + b)
##     }
## <bytecode: 0x1484ea790>
## <environment: 0x12e075aa8>
\end{verbatim}

\begin{Shaded}
\begin{Highlighting}[]
\FunctionTok{ls}\NormalTok{(}\FunctionTok{environment}\NormalTok{(nLL))}
\end{Highlighting}
\end{Shaded}

\begin{verbatim}
## [1] "data"   "fixed"  "params"
\end{verbatim}

\hypertarget{estimating-parameters}{%
\subsubsection{Estimating Parameters}\label{estimating-parameters}}

\begin{Shaded}
\begin{Highlighting}[]
\FunctionTok{optim}\NormalTok{(}\FunctionTok{c}\NormalTok{(}\AttributeTok{mu =} \DecValTok{0}\NormalTok{, }\AttributeTok{sigma =} \DecValTok{1}\NormalTok{), nLL)}\SpecialCharTok{$}\NormalTok{par}
\end{Highlighting}
\end{Shaded}

\begin{verbatim}
##       mu    sigma 
## 1.218239 1.787343
\end{verbatim}

Fixing σ = 2

\begin{Shaded}
\begin{Highlighting}[]
\NormalTok{nLL }\OtherTok{\textless{}{-}} \FunctionTok{make.NegLogLik}\NormalTok{(normals, }\FunctionTok{c}\NormalTok{(}\ConstantTok{FALSE}\NormalTok{, }\DecValTok{2}\NormalTok{))}
\FunctionTok{optimize}\NormalTok{(nLL, }\FunctionTok{c}\NormalTok{(}\SpecialCharTok{{-}}\DecValTok{1}\NormalTok{, }\DecValTok{3}\NormalTok{))}\SpecialCharTok{$}\NormalTok{minimum}
\end{Highlighting}
\end{Shaded}

\begin{verbatim}
## [1] 1.217775
\end{verbatim}

Fixing μ = 1

\begin{Shaded}
\begin{Highlighting}[]
\NormalTok{nLL }\OtherTok{\textless{}{-}} \FunctionTok{make.NegLogLik}\NormalTok{(normals, }\FunctionTok{c}\NormalTok{(}\DecValTok{1}\NormalTok{, }\ConstantTok{FALSE}\NormalTok{))}
\FunctionTok{optimize}\NormalTok{(nLL, }\FunctionTok{c}\NormalTok{(}\FloatTok{1e{-}6}\NormalTok{, }\DecValTok{10}\NormalTok{))}\SpecialCharTok{$}\NormalTok{minimum}
\end{Highlighting}
\end{Shaded}

\begin{verbatim}
## [1] 1.800596
\end{verbatim}

\hypertarget{plotting-the-likelihood}{%
\subsubsection{Plotting the Likelihood}\label{plotting-the-likelihood}}

\begin{Shaded}
\begin{Highlighting}[]
\NormalTok{nLL }\OtherTok{\textless{}{-}} \FunctionTok{make.NegLogLik}\NormalTok{(normals, }\FunctionTok{c}\NormalTok{(}\DecValTok{1}\NormalTok{, }\ConstantTok{FALSE}\NormalTok{)) }
\NormalTok{x }\OtherTok{\textless{}{-}} \FunctionTok{seq}\NormalTok{(}\FloatTok{1.7}\NormalTok{, }\FloatTok{1.9}\NormalTok{, }\AttributeTok{len =} \DecValTok{100}\NormalTok{)}
\NormalTok{y }\OtherTok{\textless{}{-}} \FunctionTok{sapply}\NormalTok{(x, nLL)}
\FunctionTok{plot}\NormalTok{(x, }\FunctionTok{exp}\NormalTok{(}\SpecialCharTok{{-}}\NormalTok{(y }\SpecialCharTok{{-}} \FunctionTok{min}\NormalTok{(y))), }\AttributeTok{type =} \StringTok{"l"}\NormalTok{)}
\end{Highlighting}
\end{Shaded}

\includegraphics{4.-Scoping-Rules_files/figure-latex/unnamed-chunk-24-1.pdf}

\begin{Shaded}
\begin{Highlighting}[]
\NormalTok{nLL }\OtherTok{\textless{}{-}} \FunctionTok{make.NegLogLik}\NormalTok{(normals, }\FunctionTok{c}\NormalTok{(}\ConstantTok{FALSE}\NormalTok{, }\DecValTok{2}\NormalTok{))}
\NormalTok{x }\OtherTok{\textless{}{-}} \FunctionTok{seq}\NormalTok{(}\FloatTok{0.5}\NormalTok{, }\FloatTok{1.5}\NormalTok{, }\AttributeTok{len =} \DecValTok{100}\NormalTok{)}
\NormalTok{y }\OtherTok{\textless{}{-}} \FunctionTok{sapply}\NormalTok{(x, nLL)}
\FunctionTok{plot}\NormalTok{(x, }\FunctionTok{exp}\NormalTok{(}\SpecialCharTok{{-}}\NormalTok{(y }\SpecialCharTok{{-}} \FunctionTok{min}\NormalTok{(y))), }\AttributeTok{type =} \StringTok{"l"}\NormalTok{)}
\end{Highlighting}
\end{Shaded}

\includegraphics{4.-Scoping-Rules_files/figure-latex/unnamed-chunk-25-1.pdf}

\hypertarget{lexical-scoping-summary}{%
\subsubsection{Lexical Scoping Summary}\label{lexical-scoping-summary}}

· Objective functions can be ``built'' which contain all of the
necessary data for evaluating the function · No need to carry around
long argument lists --- useful for interactive and exploratory work. ·
Code can be simplified and cleand up · Reference: Robert Gentleman and
Ross Ihaka (2000). ``Lexical Scope and Statistical Computing,'' JCGS, 9,
491--508.

\end{document}
